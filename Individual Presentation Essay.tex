\documentclass{article}
\usepackage[margin=1in]{geometry}
\title{ProfCom: Individual Presentation Script}
\author{Christopher Sprague}
\date{April 2014}
\pagestyle{empty}
\begin{document}
\maketitle
\thispagestyle{empty}
\paragraph{Slide 1: Intro} ~\\

Hello, everyone! My name is Christopher Sprague, and my presentation for 
you all today will be on the topic of one of my personal projects, an 
Android app I've named \emph{D2B-Android}. The premise of the app, which I will
go over in more depth shortly, is primarily to take user input in the
form of numbers written in standard decimal format, and convert the provided 
number into its corresponding binary value. As you'll notice, this is an 
open source project published on my GitHub, which you may access using the 
link shown and browse through at your leisure. And, for the record, the script 
I'm reading from happens to be written in \TeX{}, and is also available to look
over within the aforementioned repo. If you were curious, I am also creating
my SRS (Software Requirements Specification) for this app.
\paragraph{Slide 2: Inspiration \& Background} ~\\

All projects and developments stem from some idea or motivation. For many of
the projects I have developed recently, I opted to vocalize this inspiration
not just for others to see, but for my own reasons, such as
when I would look back at the work I had done for these projects, and this
one is no exception. Obviously, I took on this project to learn a little bit
more within the realm of mobile development. Furthermore, the Computer Science
field demands high levels of cooperation and ability to work in group settings;
I took up this project with two of my friends, Eric Falkenberg and Kamil Bynoe.
The more technical aspects of my reasons for taking up development included
further practice with version control using Git and GitHub, as well as working
out the conversion logic \emph{without} the use of built-in functionality.
And, obviously, the project has helped put my programming skills into action -
the conversion logic was originally written in C++ (which also happens to be
on my GitHub,) Android development is Java heavy to say the least, and
I've also begun to scratch the surface with writing in \TeX{}.
\paragraph{Slide 3: Technical Details} ~\\

Moving onto some of the more technical details... This is obviously an app
designed to be used on Android devices. The app was developed using Eclipse's
Android development environment and tools, which is a free plugin for Eclipse
you can find through a quick search for "Android Development." The program
is tested on 4.4 which corresponds to API Level 19, commonly referred to as
\emph{KitKat}. Some conventions mandate testing on the higher/highest
available build, while keeping the minimum SDK lower at the same time. In this 
case my minimum and target SDK is 16, or 4.1.x, \emph{Jelly Bean}. As stated
earlier, the project uses Git version control through GitHub. The app has not
been tested on a live device; instead, I do testing using an Android Virtual
Device (AVD for short) which is available through the Android Dev tools in
Eclipse.
\newpage
\paragraph{Slide 4: Technical Details (continued)} ~\\

Continuing on our technical details discussion... It might be a bit difficult
to make out, but the material on the bottom left is part of the Android 
Manifest xml file; the bottom right is actually a block of code from the
decimal to binary conversion logic written in C++ (a relatively quick
translation into Java from there,) specifically detailing the routine to
fetch and format what our new binary number is going to look like. At the top
right is the p2 function (again, from the C++ program,) which is arguably
the most important functionality within the conversion logic, which finds
the next highest power of two which does exceed the remaining value to convert.
Again, the code is very similar between C++ and Java, but if you're really
that interested in what it looks like in Java, it is available in the
\emph{D2B-Android} project on GitHub in the file named \emph{D2BConversionLogic.Java}.
As I briefly mentioned earlier, while producing this conversion logic
by myself without the use of any external resources could be considered
noteworthy or even commendable, someone with a much shorter patience, or
someone who is simply more focused on producing in more realistic styles
and shorter timeframes may have chosen to use Java's built-in functionality
for this base conversion logic, as shown here. Nevertheless, the fact that this
functionality is readily available can make it easy in the future to add new
features implementing new conversions, beyond simply base ten to base two
(that is, \emph{decimal to binary}!)
\paragraph{Slide 5: User Interface} ~\\

Here are some screenshots of the app in action. The screenshots are from Mac's
screen capture feature, are pictures of the AVD in action, and represent
how the app would run on a real device. In our first interface on the left,
we have a standard text field at the top in which the user will enter the
non-negative integer to be converted, as seen in figure two. Finally, when the
user is done entering their number, they may touch the \emph{Convert} button
which will take the user into a new scene with the number they entered now
in its binary form, and that's that! It should also be noted that the user may
hit the standard 'back' button at the top left (the less-than sign next to
the Android mascot, to the left of "Binary.") The user may also navigate back
to the home scene to enter more numbers to be converted using Android's standard
'return' functionality using the curved arrow button at the bottom left. It is
true that this build of the user interface is painfully simple, but I will
look into future ideas and improvements in the next slide.
\paragraph{Slide 6: Looking Forward} ~\\

While it may have been feasible, I see no reason at this time to bother
trying to get this app published on any kind of app store
like the Amazon App Store or the Play Store. As stated earlier, the purpose
of this project from the very start was simply to get practice working
with Android and mobile development in general, as well as getting some
experience working in a group setting. Moreover, the app, to me, is simply
evidence of our efforts and intentions to practice programming and learn
just a little bit more about the software development field. Like many other
projects I choose to feature on my GitHub or elsewhere, I liken projects of
this caliber to a much more time-consuming and detailed style of showing my
work much like I would for any mathematical problem. Future ideas for this project
include the possiblility of converting to more bases (again, as mentioned
earlier,) and possibly even reverse conversion, like binary-to-decimal, in
which the user would provide a number in binary and the program would produce
that number in decimal (or perhaps even in other bases,) and I think these
ideas are well within reason and would be easy enough to implement. In terms
of what I've got out of this project that I will take on with me for the rest
of my work, I can say I have had more practice working together in a group,
the logistics that are involved in working in group settings like this one,
and, of course, all the new technical details and skills
I have explored developing for this project.
\paragraph{Slide 7: Questions, Comments, Concerns} ~\\

On that note, my end of the presentation is over, though I would be more than
happy to entertain any comments or questions anyone might have at this point.

\end{document}
